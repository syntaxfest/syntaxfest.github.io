%File SyntaxFest2019.tex
\documentclass[11pt]{article}

\usepackage{SyntaxFest2019}
\usepackage{url}
\usepackage{latexsym}
\usepackage{mathptmx}
\usepackage{expex}
\usepackage[T1]{fontenc}
\usepackage{pinyin}

\usepackage{CJKutf8}
\newcommand{\Chinese}[1]{\begin{CJK*}{UTF8}{gbsn}#1\end{CJK*}}


\usepackage{enumitem}
\setlist[itemize]{noitemsep, label={\large\textbullet}}
% You can expand the titlebox if you need extra space
% to show all the authors. Please do not make the titlebox
% smaller than 5cm (the original size); we will check this
% in the camera-ready version and ask you to change it back.


%@@marie: can't succeed in adding a page foot

%\usepackage[footsepline,plainfootsepline]{scrlayer-scrpage}
%\setkomafont{pagefoot}{\scriptsize\itshape}
%\cfoot*{Blablabla}

%% \usepackage{fancyhdr}
%% \pagestyle{fancy}
%% \renewcommand\headrulewidth{1pt}
%% \fancyfoot[C]{SyntaxFest - August 26-30 2019 - Paris}

    %% \usepackage{fancyhdr}
    %% \fancyhead{}% efface le contenu de l'en-tete
    %% \fancyfoot{}% efface le contenu du pied de page
    %% \fancyhead[RO,LE]{\textbf{2009-2010}}
    %% \fancyhead[LO,RE]{\textbf{\nouppercase{\leftmark}}}
    %% \rfoot{\textit{claire Latex}}% pied de page en bas à droite sur la premiere page seulement
    %% \cfoot{\thepage}
    %% \pagestyle{fancy}

\title{Invited Talk\\
  {\small Thursday 29th August 2019}\\
  Quantitative Computational Syntax:\\ dependencies, intervention effects and word embeddings}

\author{Paola Merlo\\
  University of Geneva}
%\\
%  \texttt{email@domain} \\


%\date{}



\begin{document}
\maketitle
\begin{abstract}
In the  computational study  of intelligent  behaviour, the  domain of
language is distinguished by the complexity of the representations and
the vast amounts  of quantitative text-driven  data. In this talk,  I will
let these two  aspects of the study of language  inform each other and
will  discuss  current  work   investigating  whether  the  notion  of
similarity  in  the intervention  theory  of  locality is  related  to
current notions of similarity in word embedding space.

Despite   their  practical   success   and  impressive   performances,
neural-network-based and  distributed semantics techniques  have often
been criticized as  they remain fundamentally opaque  and difficult to
interpret.   Several  recent  pieces  of work  have  investigated  the
linguistic abilities of these representations, and shown that they can
capture long agreement  and thus hierarchical notions.   In this vein,
we  study another  core,  defining and  more  challenging property  of
language:  the ability  to  construe  long-distance dependencies.   We
present  results that  show that  word embeddings  and the  similarity
spaces  they define  do  not correlate  with  experimental results  on
intervention similarity  in long-distance dependencies.  These results
show that the linguistic  encoding in distributed representations does
not appear to be human-like, and it also brings evidence to the debate
on narrow  or broad definitions  of similarity in syntax  and sentence
      processing.
\end{abstract}

\vspace{4mm}
\begin{shortbio}
  Paola Merlo  is associate professor  in the Linguistics  department of
the University  of Geneva.  She  is the head of  the interdisciplinary
research  group Computational  Learning and  Computational Linguistics
(CLCL).   The  group  is  concerned  with  interdisciplinary  research
combining  linguistic  modelling  with  machine  learning  techniques.
Prof.  Merlo  has been editor of  Computational Linguistics, published
by  MIT  Press  and  a  member  of  the  executive  committee  of  the
ACL. Prof. Merlo holds a  doctorate in  Computational Linguistics  from the
University of Maryland,  and  has been  associate research fellow
at the  University of Pennsylvania,  and visiting scholar  at Rutgers,
Edinburgh, Stanford and Uppsala.
\end{shortbio}

\end{document}
